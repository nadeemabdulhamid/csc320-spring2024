\documentclass[12pt]{article}
	
\title{CSC 320 - Homework 4}
\author{}
\date{Due: February 5, 10pm}


\usepackage[margin=1in]{geometry}		% For setting margins
\usepackage{amsmath}				% For Math
\usepackage{fancyhdr}				% For fancy header/footer
\usepackage{graphicx}				% For including figure/image
\usepackage{cancel}					% To use the slash to cancel out stuff in work

\usepackage{algorithm,caption}
\usepackage{algpseudocodex}
% docs: https://ctan.math.washington.edu/tex-archive/macros/latex/contrib/algpseudocodex/algpseudocodex.pdf


%%%%%%%%%%%%%%%%%%%%%%
% Set up fancy header/footer
% taken from https://www.overleaf.com/latex/templates/homework-template/yvgnmrbywwnp
\makeatletter    % for \@ in \@title
\pagestyle{fancy}
\fancyhead[LO,L]{\@author}
\fancyhead[CO,C]{\@title}
\fancyhead[RO,R]{\@date}
\fancyfoot[LO,L]{}
\fancyfoot[CO,C]{\thepage}
\fancyfoot[RO,R]{}
\renewcommand{\headrulewidth}{0.4pt}
\renewcommand{\footrulewidth}{0.4pt}
\makeatother    % restore
%%%%%%%%%%%%%%%%%%%%%%


%%%%%%%%%%%%%%%%%%%%%%
% from: https://tex.stackexchange.com/questions/14667/does-latex-define-a-semantic-equivalent-of-textbf
\makeatletter
\newcommand{\strong}[1]{\@strong{#1}}
\newcommand{\@@strong}[1]{\textbf{\let\@strong\@@@strong#1}}
\newcommand{\@@@strong}[1]{\textnormal{\let\@strong\@@strong#1}}
\let\@strong\@@strong
\makeatother
%%%%%%%%%%%%%%%%%%%%%%

\renewcommand*\contentsname{\small Overview}

\begin{document}
%{\small \tableofcontents}

\section{Independent Set on Paths}

Let $G = (V, E)$ be an undirected graph with $n$ nodes. A subset of the nodes is called an \textbf{independent set} if no two of them are joined by an edge. Finding large independent sets is difficult in general; but here we'll see that it can be done efficiently if the graph is ``simple'' enough.

Call a graph $G = (V, E)$ a \textbf{path} if its nodes can be written as $v_1, v_2, \ldots, v_n$, with an edge between $v_i$ and $v_j$ if and only if the numbers $i$ and $j$ differ by exactly 1. With each node $v_i$, we associate a positive integer \emph{weight} $w_i$. 

The goal in this exercise is to solve the following problem:

\begin{quotation}\it
    Find an independent set in a path $G$ whose total weight is as large as possible.
\end{quotation}

\begin{itemize}
    \item Draw the five-node path with node weights (in order): 1, 8, 6, 3, 6.
    \item Give an example to show that the following algorithm \emph{does not} always find an independent set of maximum total weight:
    
    \begin{verbatim}
        (The "heaviest-first" greedy algorithm)
          Start with S equal to the empty set
          While some node remains in G:
            Pick a node v_i of maximum weight
            Add v_i to S
            Delete v_i and its neighbors from G

          Return S
    \end{verbatim}

    \item Give an example to show that the following algorithm also \emph{does not} always find an independent set of maximum total weight.
    
    \begin{verbatim}
        Let S_1 be the set of all v_i where i is an odd number.
        Let S_2 be the set of all v_i where i is an even number.
        (Note that both S_1 and S_2 are independent sets.)
        Determine the greater of the sum of weights in S_1 or S_2 
            and return that set.
    \end{verbatim}

    \item Describe an algorithm that takes an $n$-node path $G$ with weights and returns an independent set of maximum total weight. The running time should be polynomial in $n$, independent of the values of the weights. (Try, if you can, to define and solve a recurrence relation for the running time of your algorithm.)
\end{itemize}


\section{Magic Squares}

\textit{You may use the Internet to search for references, explanations, proposed solutions to this exercise. Explain answers to the questions below in your own words.}

A magic square  of order $n$ is an arrangement of the integers from 1 to $n^2$ in an $n \times n$ matrix, with each number occurring exactly once, so that each row, each column, and each main diagonal has the same sum.

\begin{itemize}
    \item For a magic square of order $n$, what does the sum of each row equal? Prove that. 
    \item Describe a backtracking algorithm (in the form of how backtracking algorithms are described in our textbook and how we've covered in class) to generate \emph{a} magic square.
\end{itemize}



\section{Addition Chain}

Exercise 3(a), Chapter 2 (page 94) in the textbook.



\end{document}
